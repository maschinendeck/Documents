\documentclass[parskip=half]{scrreprt}
\usepackage[ngerman]{babel} 
\usepackage[utf8]{inputenc} 
\usepackage[T1]{fontenc} 
\usepackage[juratotoc]{scrjura}
\usepackage{enumerate}


\newcommand{\associationName}{Maschinendeck} %Vereinsname (ohne e.V.)
\newcommand{\issuedOnDate}{23.7.2014} %wann beschlossen


 
\begin{document}
\KOMAoptions{ref=nosentence}
	
\title{Geschäftsordnung des Vorstands des \associationName{} e.V.} 
\subtitle{beschlossen am \issuedOnDate{}}
\author{} 
\date{} 
\maketitle
 
\tableofcontents
\newpage

\begin{contract}
	                                        
\Paragraph{title={Versammlungsordnung}}

\begin{enumerate}
\item Der Vorstand soll mindestens einmal im Quartal tagen.
\item Der Vorstand wählt aus seinen Reihen einen Protokollführer, der den Ablauf der Vorstandssitzung protokolliert.                                         
\item Über den Verlauf der Vorstandssitzungen ist eine Niederschrift anzufertigen, die vom allen Anwesenden zu unterzeichnen ist. Die Niederschrift ist innerhalb einer Woche den Mitgliedern schriftlich oder per E-Mail zur Verfügung zu stellen. Erfolgt nach der Veröffentlichung der Niederschrift innerhalb von vier Wochen kein Einspruch, gilt diese als genehmigt. 
\end{enumerate}
                                        
\Paragraph{title={Berichtspflicht des Vorstands}}

Mit dem Ablauf des Geschäftsjahres:
\begin{enumerate}
\item erstellt der Vorstand den Geschäftsbericht 
\item stellt der Schatzmeister unverzüglich die Abrechnung sowie die Vermögensübersicht und sonstige Unterlagen von wirtschaftlichem Belang dem Rechnungsprüfer des Vereins zur Prüfung zur Verfügung. 
\end{enumerate}
                                        
\Paragraph{title={Beiräte}}

Der Vorstand kann fachliche Beiräte oder wissenschaftliche Beiräte einrichten, die für den Verein beratend und unterstützend tätig werden; in die Beiräte können auch Nicht-Mitglieder berufen werden.

\Paragraph{title={Inkrafttreten}}
\begin{enumerate}
\item Diese Geschäftsordnung wurde durch den Vorstand am \issuedOnDate{} beschlossen.
\item Diese Geschäftsordnung ersetzt alle vorher beschlossenen Geschäftsordnungen.
\end{enumerate}

\end{contract}
 
\end{document}