\documentclass[parskip=half]{scrreprt}
\usepackage[ngerman]{babel} 
\usepackage[utf8]{inputenc} 
\usepackage[T1]{fontenc} 
\usepackage[juratotoc]{scrjura}
\usepackage{enumerate}


\newcommand{\associationName}{Maschinendeck} %Vereinsname (ohne e.V.)
\newcommand{\issuedOnDate}{\today{}} %wann beschlossen


 
\begin{document}
\KOMAoptions{ref=nosentence}
	
\title{Geschäftsordnung der Mitgliederversammlung des \associationName{} e.V.} 
\subtitle{beschlossen am \issuedOnDate{}}
\author{} 
\date{} 
\maketitle
 
\tableofcontents

\newpage

In Ausfüllung und Ergänzung des von der Satzung des \associationName{} e.V. vorgegebenen Rahmens wird folgende Geschäftsordnung erlassen: 


\begin{contract}

\Paragraph{title={Protokollführer}}

Die Mitglieder wählen aus ihren Reihen einen Protokollführer.

Über den Verlauf der Mitgliederversammlungen ist eine Niederschrift anzufertigen, die vom Versammlungsleiter und vom Protokollführer zu unterzeichnen ist.
Diese Niederschrift ist auf Anfrage beim Vorstand einsehbar. Erfolgt innerhalb von vier Wochen nach Unterzeichnung der Niederschrift kein Einspruch gilt diese als genehmigt.

Die Niederschrift soll folgende Angaben enthalten:
\begin{enumerate}
	\item Ort und Tag der Versammlung
	\item Name des Versammlungsleiters und Protokollführers
	\item die Zahl der erschienen Mitglieder
	\item Angaben zu den gefassten Beschlüssen mit genauen Abstimmungsergebnissen e) die erforderlichen Unterschriften
\end{enumerate}

\Paragraph{title={Rechnungsprüfer}}

Die Mitgliederversammlung kann, jeweils für die Dauer von einem Geschäftsjahr, zwei Kassenprüfer wählen, die nicht Mitglied des Vorstandes sind. Eine Wiederwahl ist zulässig.

Der Rechnungsprüfer prüft die Kassen- und Rechnungsführung des Vorstandes nach Ablauf eines jeden Geschäftsjahres und berichtet darüber auf der ordentlichen Mitgliederversammlung.

Die Tätigkeit ist ehrenamtlich.

Der Rechnungsprüfer kann nach eigenem Ermessen unter betriebswirtschaftlicher Beachtung der Finanzkraft des Vereins zur Rechnungsprüfung vereidigte Wirtschaftsprüfer oder Steuerberater hinzuziehen, welche gegebenenfalls die Kassen- und Rechnungsprüfung zu testieren haben. Eine Verpflichtung dazu besteht nur dann, wenn die Mitgliederversammlung dies ausdrücklich für den Einzelfall beschließt.

\Paragraph{title={Beschlussfassung und Wahlen}}

Die Abstimmung erfolgt offen, soweit keines der anwesenden Mitgliedern eine geheime Abstimmung verlangt.

Die offene Abstimmung erfolgt durch Handzeichen.

Die Wahl des Vorstand und der Rechnungsprüfer erfolgt grundsätzlich geheim. 
                                        
\Paragraph{title={Ablauf der Mitgliederversammlung}}

\begin{enumerate}
\item Der Versammlungsleiter eröffnet die Sitzung und begrüßt die anwesenden Mitglieder.
\item Der Versammlungsleiter beantragt die Feststellung der Beschlussfähigkeit.
\item Die Mitgliederversammlung bestimmt einen Protokollführer.
\item Der Versammlungsleiter beantragt die Genehmigung der Tagesordnung. Hier soll zusätzlich darüber befunden werden, ob über nachträglich gestellte Anträge beschossen werden kann.
\item Die Mitgliederversammlung tritt in die Tagesordnung ein. Jeder Beschlussfassung soll eine Aussprache vorangestellt sein.
\item Sofern ein Tagesordnungspunkt „Verschiedenes“ existiert, soll dieser nur für Informationen und Ankündigungen verwendet werden. Innerhalb dieses Tagesordnungspunktes sollen keine Beschlüsse gefasst werden.
\end{enumerate}
                                                                          
\Paragraph{title={Inkrafttreten}}
                         
Diese Geschäftsordnung wurde durch die Mitgliederversammlung vom \issuedOnDate{} beschlossen und tritt mit sofortiger Wirkung in Kraft.

Diese Geschäftsordnung ersetzt alle vorher beschlossenen Geschäftsordnungen.

\end{contract}
 
\end{document}