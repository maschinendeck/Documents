\documentclass[parskip=half]{scrreprt}
\usepackage[ngerman]{babel} 
\usepackage[utf8]{inputenc} 
\usepackage[T1]{fontenc} 
\usepackage[juratotoc]{scrjura}
\usepackage{enumerate}

\newcommand{\associationName}{Maschinendeck} %Vereinsname (ohne e.V.)
\newcommand{\issuedOnDate}{\today{}} %wann beschlossen

\begin{document}
\KOMAoptions{ref=nosentence}
		
\title{Satzung des \associationName{} e.V.} 
\subtitle{beschlossen am \issuedOnDate{}} 
\author{} 
\date{} 
\maketitle
 
\tableofcontents
 
\begin{contract}
\Paragraph{title={Name, Sitz, Geschäftsjahr}}

Der Verein führt den Namen \associationName{} und soll in das Vereinsregister eingetragen werden; nach der Eintragung führt er den Zusatz \shorthandoff{"}"e.V."\shorthandon{"}.
                           
Der Verein hat seinen Sitz in Trier.

Das Geschäftsjahr des Vereins ist das Kalenderjahr.

\Paragraph{title={Zweck des Vereins}}
                                        
Zweck des Vereins ist die Förderung der Bildung und Volksbildung auf dem Gebiet der Informationstechnologien, des Informationsrechts und verwandten Themen sowie des kreativen Umgangs mit diesen.
                      
Der Verein verfolgt ausschließlich und unmittelbar gemeinnützige Zwecke im Sinne des Abschnittes \shorthandoff{"}"Steuerbegünstigte Zwecke"\shorthandon{"} der Abgabenordnung.

Der Satzungszweck wird insbesondere verwirklicht durch:
\begin{enumerate}
\item Durchführung von öffentlichen, entgeltfreien Veranstaltungen für Computersicherheit, Informationsrecht und kreativen Umgang mit neuen Technologien und deren Anwendungen.
\item Förderung von Forschung, Entwicklung und Aufklärung im Bereich der Informationstechnologien.
\item Förderung der Allgemeinbildung der Bevölkerung im Umgang mit neuen Technologien. 
\end{enumerate}

\Paragraph{title={Selbstlosigkeit}}
                                        
Der Verein ist selbstlos tätig; er verfolgt nicht in erster Linie eigenwirtschaftliche Zwecke.

Mittel des Vereins dürfen nur für die satzungsgemäßen Zwecke verwendet werden. Die Mitglieder erhalten keine Gewinnanteile und in ihrer Eigenschaft als Mitglieder auch keine sonstigen Zuwendungen aus Mitteln des Vereins. Es darf keine Person durch Ausgaben, die dem Zweck des Vereins fremd sind, oder durch unverhältnismäßig hohe Vergütungen begünstigt werden.

Alle Inhaber von Vereinsämtern sind ehrenamtlich tätig.

\Paragraph{title={Erwerb der Mitgliedschaft}}
                                       
Der Verein hat ordentliche, Ehren- und Fördermitglieder

Ordentliche Mitglieder können ausschließlich natürliche Personen werden.                                        

Die Mitgliederversammlung kann Personen, die sich durch besondere Verdienste im Sinne des Vereins oder die von ihm verfolgten satzungsgemäßen Zwecke hervorgetan haben, zu Ehrenmitgliedern ernennen. Ehrenmitglieder haben alle Rechte eines ordentlichen Mitglieds. Sie sind von Beitragsleistungen befreit.  

Fördermitglied kann jede natürliche oder juristische Person werden.                                       

Der Vorstand entscheidet auf schriftlichen Antrag des potentiellen Mitglieds über die Aufnahme. Der Beschluss wird dem Antragsteller schriftlich oder per E-Mail mitgeteilt.                                         

Gegen den ablehnenden Bescheid des Vorstands kann der Antragsteller Beschwerde einlegen, die binnen eines Monats ab Zugang der Ablehnung schriftlich beim Vorstand einzureichen ist. Über die Beschwerde entscheidet die Mitgliederversammlung nach demselben Verfahren wie bei Ausschluss eines Mitglieds.                                        
Die Mitgliedschaft beginnt nach positivem Aufnahmebescheid mit dem Eingang des Aufnahmebeitrags und des ersten Mitgliedsbeitrags.

Im Falle nicht fristgerechter Entrichtung der Beiträge ruht die Mitgliedschaft.

\Paragraph{title={Beendigung der Mitgliedschaft}}

Die Mitgliedschaft endet 
\begin{enumerate}                                        
\item bei natürlichen Personen mit deren Tod.                                         
\item nach Austrittserklärung eines Mitglieds. Die Austrittserklärung erfordert die Schriftform und muss gegenüber dem Vorstand mit einer Frist von zwei Wochen zum Ende des Monats eingereicht werden.                                        
\item bei Mitgliedern, die sich nach schriftlicher Mahnung mehr als sechs Monate mit Mitgliedsbeiträgen im Verzug befinden.                                   
\item durch Ausschluss aus dem Verein.
\end{enumerate}

\Paragraph{title={Mitgliedsbeiträge}}
   
Der Verein erhebt einen Aufnahmebeitrag und einen regelmäßigen Mitgliedsbeitrag, die im Voraus zu entrichten sind. Näheres regelt eine von der Mitgliederversammlung zu beschließende Beitragsordnung.

Bei Beendigung der Mitgliedschaft, gleich aus welchem Grund, erlöschen alle Ansprüche aus dem Mitgliedsverhältnis. Eine Rückerstattung von Beiträgen, Spenden oder sonstigen Unterstützungsleistungen ist grundsätzlich ausgeschlossen. Der Anspruch des Vereins auf offene Beitragsforderungen bleibt hiervon unberührt.

\Paragraph{title={Organe}}
Die Organe des Vereins sind 
\begin{enumerate}      
\item die Mitgliederversammlung 
\item der Vorstand
\end{enumerate}

\Paragraph{title={Die Mitgliederversammlung}}
                                        
Die Mitgliederversammlung besteht aus den Mitgliedern des Vereins.                                        

Die ordentliche Mitgliederversammlung wird einmal jährlich im ersten Halbjahr vom Vorstand einberufen.                                        

Es können außerordentliche Mitgliederversammlungen entweder auf Beschluss des Vorstands oder auf Verlangen eines Fünftels der Mitglieder einberufen werden.                                       
Die Einladung zur Mitgliederversammlung ist den Mitgliedern schriftlich oder per E- Mail unter Angabe von Ort, Zeit und Tagesordnung mindestens vier Wochen vorher zuzustellen. Die Einladung erfolgt an die letzte vom Mitglied bekannt gegebene Adresse. \label{labA}                                      

Mitglieder können zu den bestehenden Tagesordnungspunkten weitere Anträge stellen, wenn sie diese dem Vorstand spätestens zwei Wochen vor dem anberaumten Termin schriftlich oder per E-Mail zur Bekanntgabe mitteilen. Die Mitgliederversammlung beschließt über die Zulassung der nachträglichen Anträge zur Beschlussfassung. \label{labB}

Eine Vertretung eines Mitglieds durch ein anderes ist möglich, wenn die Vertretungsbefugnis schriftlich nachgewiesen wird.                                         
Jede ordnungsgemäß einberufene Mitgliederversammlung ist unabhängig von der Zahl der erschienenen Mitglieder beschlussfähig.                                         
Die Mitgliederversammlung bestellt einen Versammlungsleiter. 
                                               
Die Mitglieder wählen aus ihren Reihen einen Protokollführer. 

Über den Verlauf der Mitgliederversammlungen ist eine Niederschrift anzufertigen, die vom Versammlungsleiter und vom Protokollführer zu unterzeichnen ist. Diese Niederschrift ist auf Anfrage beim Vorstand einsehbar. Erfolgt innerhalb von vier Wochen nach Unterzeichnung der Niederschrift kein Einspruch gilt diese als genehmigt. 

Die Niederschrift soll folgende Angaben enthalten:
\begin{enumerate}
\item Ort und Tag der Versammlung 
\item Name des Versammlungsleiters und Protokollführers 
\item die Zahl der erschienen Mitglieder
\item Angaben zu den gefassten Beschlüssen mit genauen Abstimmungsergebnissen 
\item die erforderlichen Unterschriften 
\end{enumerate}

Jedes Mitglied, dessen Mitgliedschaft nicht ruht, ist stimmberechtigt.
 
Fördermitglieder sind von diesen Regelungen ausgenommen. Sie können aber in der Mitgliederversammlung Anträge stellen und werden ebenso umfassend wie ordentliche Mitglieder und Ehrenmitglieder über alle Beschlüsse des Vereins informiert.

\Paragraph{title={Zuständigkeiten der Mitgliederversammlung}} 
                                        
Die Mitgliederversammlung 
\begin{enumerate}                                                                                         
\item wählt und kontrolliert den Vorstand. 
\item prüft und genehmigt die Jahresabschlussrechnung des Schatzmeisters und erteilt die Entlastung des Vorstands. 
\item entscheidet in allen Fällen, in denen nicht die Zuständigkeit eines anderen Organs bestimmt ist. 
\item trifft Mehrheitsentscheidungen mit der einfachen Mehrheit der teilnehmenden Mitglieder. 
\item kann den Vereinszweck mit der Zustimmung aller teilnehmenden Mitglieder ändern. Der Änderungsantrag muss gemäß \ref{labA} erfolgen. Weiter wird bestimmt, dass \ref{labB} für Zweckänderungen keine Anwendung findet. Zweckänderungen können somit nicht durch Nachtrag zur Tagesordunung beschlossen werden. 
\item kann die Vereinssatzung mit Zustimmung von drei Vierteln der teilnehmenden Mitglieder ändern. 
\item gibt sich eine Geschäftsordnung.
\end{enumerate}

\Paragraph{title={Der Vorstand}}
                                                                                         
Der Vorstand trifft seine Beschlüsse auf Sitzungen.

Der Vorstand ist beschlussfähig, wenn mindestens zwei Vorstandsmitglieder anwesend sind.

Beschlüsse im Vorstand werden mit einfacher Mehrheit gefasst.

Bei Ausscheiden eines Vorstandsmitglieds kann durch den Vorstand für die verbleibende Amtszeit ein Stellvertreter bestellt werden.

Der Verein wird gerichtlich und außergerichtlich durch die Vorstandsmitglieder vertreten. Jeder ist alleinvertretungsberechtigt.

Im Innenverhältnis wird bestimmt, dass der Vorstand in wichtigen Dingen gemeinsam beschließt.

Der Vorstand wird von der Mitgliederversammlung für die Dauer von einem Jahr bestellt, er bleibt jedoch bis zur Bestellung eines neuen Vorstandes im Amt. Die Wiederwahl ist zulässig.

Vorstandsmitglied kann nur werden, wer mindestens ein Jahr Vereinsmitglied ist.

Der Vorstand besteht aus drei gleichberechtigten Personen. Die Mitgliederversammlung bestellt davon eine als Schatzmeister.

\Paragraph{title={Zuständigkeiten des Vorstands}}
                                                                                         
Der Vorstand führt die Geschäfte des Vereins und fasst die erforderlichen Beschlüsse.

Er ist zu rechtsgeschäftlichen Verpflichtungen zu Lasten des Vereins bis zu einer Höhe von EUR 523,42 ermächtigt. Diese Bestimmung betrifft das Innenverhältnis.

Bei höheren Ausgaben sind die Mitglieder vorher schriftlich oder per E-Mail zu unterrichten. Erfolg innerhalb von zwei Wochen kein Widerspruch gilt die Ausgabe als genemigt. Ansonsten entscheidet die Mitgliedervesammlung.

In dringenden, keinen Aufschub duldenden Dingen kann der Vorstand mit der Zustimmung aller Vorstandsmitglieder über diese Befugnisse hinaus handeln. Diese Bestimmung betrifft das Innenverhältnis. Er ist verpflichtet die Mitglieder hierüber unverzüglich zu informieren.

Der Vorstand gibt sich eine Geschäftsordnung. Diese ist den Mitgliedern innerhalb einer Woche schriftlich oder per E-Mail zur Verfügung zu stellen.
                                                                                 
\Paragraph{title={Ausschluss von Mitgliedern}}
                                                                                         
Der Vorstand kann mit einfacher Mehrheit ein Mitglied auf Antrag ausschließen.

Gegen diesen Ausschluss kann schriftlich Widerspruch eingelegt werden.

Ein Widerspruch führt zu einer Überprüfung des Ausschlusses durch die Mitgliederversammlung. Die einfache Mehrheit kann den Ausschluss ablehnen.

Bis zur Entscheidung der Mitgliederversammlung ruht die Mitgliedschaft.
                                                                                 
\Paragraph{title={Auflösung des Vereins}}
                                                                                         
Der Antrag auf Auflösung des Vereins kann durch den Vorstand oder ein Fünftel der Mitglieder gestellt werden.

Die Auflösung des Vereines kann nur in einer eigens zu diesem Zweck einberufenen Mitgliederversammlung mit einer Mehrheit von drei Vierteln der abgegebenen gültigen Stimmen beschlossen werden. Stimmenthaltungen bleiben außer acht.

Der Antrag auf Auflösung muss der Mitgliederversammlung spätestens vier Wochen vor ihrer Tagung vorgelegt worden sein.

Bei der Auflösung des Vereins oder bei Wegfall seines Zweckes fällt das Vereinsvermögen an die Wau Holland Stiftung, die es ausschließlich für gemeinnützige Zwecke zu verwenden hat.
                                                                                 
\Paragraph{title={Ermächtigung}}
                                        
Der Vorstand ist ermächtigt, etwaige zur Eintragung des Vereins und Anerkennung der Gemeinnützigkeit erforderliche formelle Änderungen und Ergänzungen der Satzung vorzunehmen. 
                                        
\Paragraph{title={Aufwendungsersatz}}
                                        
Werden Amtsträger, Mitglieder und Mitarbeiter des Vereins mit Tätigkeiten für den Verein beauftragt, so haben sie Anrecht auf die Erstattung ihrer Aufwendungen, sofern diese vorher dem Vorstand bekannt gegeben wurden. Die Höhe der Erstattung erfolgt bis zur Höhe der tatsächlich angefallenen Kosten, im Falle von Reisekosten bis zur Höhe der billigsten Reisemöglichkeit. Die tatsächlichen Kosten sind nachzuweisen. Abweichungen können durch den Vorstand im Einzelfall beschlossen werden. 

\end{contract}
 
\end{document}
